\documentclass{article}
\usepackage[utf8]{inputenc}
\usepackage{fullpage} % package that specifies normal margins
\usepackage{amsmath}

\title{Inverted Pendulum on a Cart Dynamics}
\author{Trey Fortmuller}
\date{June 2019}

\begin{document}

\maketitle

\section*{Introduction}

Here we derive the equations of motion for the canonical control problem of the inverted pendulum on a linear cart. They can be derived via Newtonian mechanics which is a bit complicated, or a Lagrangian approach which is more elegant. We'll walk through both, references are below.

\begin{itemize}
    \item Newtonian mechanics: https://www.youtube.com/watch?v=5qJY-ZaKSic
    \item Lagrangian mechanics: https://www.youtube.com/watch?v=7Tvo8jXlPuk
\end{itemize}

\section*{Newtonian Approach}

The state variables will be $x$, the lateral position of the cart along its track, and $theta$, the angle of the pendulum with respect to the vertical such that 0 radians is the downward resting position. We'll construct free body diagrams for the cart and for the pendulum bob.

SOME DIAGRAMS GO HERE

Now writing Newton's second law component wise for each free body diagram we get

\begin{itemize}
    \item Cart $\hat{i}$: $F - T\sin{\theta} = M\ddot{x}$
    \item Cart $\hat{j}$: The motion of the cart is constrained to the $\hat{i}$ direction
    \item Pendulum $\hat{i}$: $T\sin{\theta} = ma_{px}$
    \item Pendulum $\hat{j}$: $-mg - T\cos{\theta} = ma_{py}$
\end{itemize}

Now we can use kinematics equations to get $a_{px}$ and $a_{py}$ in terms of the state variables. The total acceleration of the pendulum is

$$ a_{pend} = a_{cart} + a_{p/c} $$

with $a_{p/c}$ the acceleration of the pendulum bob with respect to the cart.

WHAT HAPPENS RIGHT HERE WITH THIS TERM???

Because the pendulum is constraint to rotate about its pin joint in a circular motion, this equation can be rewritten:

$$ a_{pend} = \ddot{x}\hat{i} + (L\ddot{\theta}\hat{e_{\theta}} - L\dot{\theta}^2\hat{e_{r}}) $$

Now writing the $a_{p/c}$ term in terms of the $\hat{i}$ and $\hat{j}$ basis:

$$ a_{pend} = \ddot{x}\hat{i} + L\ddot{\theta}(-\cos{\theta}\hat{i} - \sin{\theta}\hat{j}) - L\dot{\theta}^{2}(-\sin{\theta}\hat{i} + \cos{\theta}\hat{j}) $$

Now we can split $a_{pend}$ into $\hat{i}$ and $\hat{j}$ components for our equations of motion. We rewrite the pendulum equations:

\begin{equation}
        \hat{i}: T\sin{\theta} = m(-L\ddot{\theta}\cos{\theta} + L\dot{\theta}^{2}\sin{\theta})
\end{equation}

\begin{equation}
    \hat{j}: -mg -T\cos{\theta} = m(\ddot{x} - L\ddot{\theta}\sin{\theta} - L\dot{\theta}^2\cos{\theta})
\end{equation}

Now we have 3 equations describing the motion of the cart pendulum system. We can eliminate $T$, the tension on the rigid pendulum rod with the following operation:

$$ (1)\cdot\cos{\theta} + (2)\cdot\sin{\theta} $$

We arrive at

\begin{equation}
    -mg\sin{\theta} = m\ddot{x}\cos{\theta} - mL\ddot{\theta}
\end{equation}

In the cart $\hat{i}$-direction equation, replace $T\sin{\theta}$ with equation $(1)$'s right hand side, we get

$$ F - m(-L\ddot{\theta}\cos{\theta} - mL\dot{\theta}^2\sin{\theta}) = M\ddot{x} $$

\begin{equation}
    F + mL\ddot{\theta}\cos{\theta} - mL\dot{\theta}^{2}\sin{\theta} = (m + M)\ddot{x}
\end{equation}

Equations (3) and (4) are the two nonlinear differential equations in the state variables that describe the motion of the cart/pendulum system. Notice $x$ doesn't appear in these equations, so the dynamics are agnostic to where on the track the cart is.

\[
\boxed{-mg\sin{\theta} = m\ddot{x}\cos{\theta} - mL\ddot{\theta}}
\]

\[
\boxed{ F + mL\ddot{\theta}\cos{\theta} - mL\dot{\theta}^{2}\sin{\theta} = (m + M)\ddot{x}}
\]

We can simulate these differential equations with Matlab or SciPy's odeint integrator in python.

\section*{Lagrangian Approach}



\end{document}

