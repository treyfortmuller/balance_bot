\documentclass{article}
\usepackage[utf8]{inputenc}
\usepackage{fullpage} % package that specifies normal margins
\usepackage{amsmath}
\usepackage{dirtytalk}

\title{Inverted Pendulum Dynamics}
\author{Trey Fortmuller}
\date{June 2019}

\begin{document}

\maketitle

\section*{Introduction}

We'll apply Newtonian mechanics to a pendulum bob fixed on a rigid rod attached to a fixed axis of rotation. We'll then include a damping (friction) term, and a motor torque term in the case of a motor acting at axis of rotation of the pendulum.

We'll start with a free body diagram of the system.

FBD GOES HERE

Newton's second law in angular form is

$$
\Sigma \tau = I \alpha
$$

If gravity is the only force acting on the pendulum bob, we arrive at a nonlinear second order ODE for the motion of the system:

$$
mgl\sin{\theta} = ml^2\ddot{\theta}
$$

With a damping force that is linear with the angular velocity of the pendulum subject to some damping factor $k$, we have:

$$
mgl\sin{\theta} - k\dot{\theta} = ml^2\ddot{\theta}
$$

Note that the units on the damping factor $k$ are $\bigg[\frac{N\cdot m}{\text{rad}/s} \bigg]$. Finally, including some motor torque acting at the axis of rotation of the pendulum, we have:

$$
mgl\sin{\theta} - k\dot{\theta} - \tau_m = ml^2\ddot{\theta}
$$

The motor torque will be the control input of the system in the case of controlling the inverted pendulum about the unstable \say{up} position. Now for the sake of simulating these dynamics using Scipy's \texttt{odeint} function, we'll convert this second order ODE into two first order ODEs by assigning an intermediate variable. The state of the system becomes:

$$
U = \begin{bmatrix}
           \theta \\
           \dot{\theta}
    \end{bmatrix}
$$

We'll assign the intermediate variable $\omega = \dot{\theta}$, the angular velocity of the pendulum bob. Our two first order ODEs are:

\begin{equation}
    \omega = \dot{\theta}
\end{equation}

\begin{equation}
    \dot{\omega} = \frac{g}{l}\sin{\theta} - \frac{k}{ml^2}\omega - \frac{1}{ml^2}\tau_m
\end{equation}

\end{document}

